\documentclass[11pt,a4paper]{article}
\usepackage[english, ngerman]{babel}
\usepackage{natbib}
\usepackage{hyperref}
\usepackage{graphicx}
\usepackage{subcaption}
\usepackage[acronym, toc, numberedsection]{glossaries}

\makenoidxglossaries
\newacronym{bs}{BS}{Radio Base Station}
\newacronym{ue}{UE}{User Equipment}

\makeatletter         
\renewcommand\maketitle{
{\raggedright
\begin{center}
{\Large \bfseries \@title}\\[2ex] 
\@author\\[1ex] 
\@date, Hochschule der Medien, Stuttgart\\[1ex]
\end{center}}} 
\makeatother

\bibliographystyle{plainnat}

\title{Energy Footprint of Mobile communications in the 21st century}
\author{Andreas Nicklaus}

\begin{document}
\maketitle

\begin{abstract}
  deutsche Version
\end{abstract}

\selectlanguage{english}

\begin{abstract}
  englische Version
\end{abstract}

\tableofcontents

\section{Introduction}\label{sec:intro}
% Vision: why and what is this topic?
Since around 2010, mobile communication has been a vital part of everyday-life for most of the western world.
In 2007, we saw the launch of the iPhone, arguably one of the most influencial inventions in the mobile market ever.
With the introduction of 4G in late 2009 came a massive increase in mobile communication over the internet.
For almost one and a half decades, more and more technology has been invented for the mobile phone market.

In addition, the Oculus Rift has ushered in the reincarnation of the idea of Virtual Reality and spacial computing, plainly meaning the usage of 3D-space as a way to distribute user interfaces.
More and more, we rely on small, wireless devices with a sleek, modern and fashionable design and people seem to keep buying in. 
With that in mind, most mobile user equipment has had one spatial constraint that has so far never been overcome: Battery life and usage breaks to recharge at a wall socket.
This paper focuses on the effectiveness of energy consumption in respect to the overall environmental impact of modern mobile communications.
To this end, the following chapter~\ref{sec:relatedwork} summarizes other work done in the field and its topics.
Chapter~\ref{sec:energyfootprint} introduces a definition of environmental footprint and its relevance to this topic.
The chapters~\ref{sec:influence} and~\ref{sec:energyconsumption} go into detail about the energy production and consumption in Germany both in general and related to mobile communications.
Chapter~\ref{sec:opportunities} then names a few opportunities for saving energy.

\section{Related Work}\label{sec:relatedwork}
% What sources have the same or similiar topic?

% 5G Energy Efficiency \cite{5GEfficiencyOverview}
% - describes the energy consumption and efficiency of base stations in detail
% - power consumption of switching between sleep mode an active state

% Dynamic gNodeB Sleep Control fo rEnergy-Conserving 5G Radio Access Network \cite{DynamicSleepModeControl}
% - outlines the changes between 4G and 5G
% - proposes sleep mode switching policies
% - comparison

% 5G Power Whitepaper \cite{powerwhitepaper}
% - insights into the power consumption of parts of base stations
% - challenges in site power construction
% - pointers to the meaning of efficient energy saving strategies

% Öffentliche Zahlen und Angaben
% - Bruttostromerzeugung bis 2022 \cite{Bruttostromerzeugung2022}
% - Umweltindikatoren \cite{Umweltindikatoren}

\section{Energy footprint}\label{sec:energyfootprint}
% What does it mean here?
\subsection{Indicators for energy footprint}\label{subsec:indicators}
% Of what does energy footprint consist?
% Categories
% Indicators
\subsection{Relevancy of indicators}\label{subsec:relevancy}
% Values for the first indicators
% TODO: First tests / findings
% What will be the topic of the rest of the chapters

\section{Influence of mobile communication on national power consumption}\label{sec:influence}
% Narrowing down the topic "environment" to "energy and power" to electricity
% Specifying Germany as the investigated region
\subsection{National averages of Germany}\label{subsec:nationalaverage}
% What is the national power consumption and production in Germany?
\subsection{Base Stations}\label{subsec:BSInfluence}
% How much do the base stations consume?
\subsection{User Equipment}\label{subsec:UEInfluence}
% TODO: How much do user equipments consume?
% What percentage of the national power consumption / production is that? 

\section{Sources of energy consumption}\label{sec:energyconsumption}
\subsection{Base Stations}\label{subsec:BSConsumption}
% What is in a base station?
% What parts are responsible for most of the power consumption
\subsection{User Equipment}\label{subsec:UEConsumption}
% TODO: What is in a user equipment?
% What parts are responsible for most of the power consumption

\section{Energy saving opportunities}\label{sec:opportunities}
% What possibilities are/will be deployable for power consumption?
\subsection{Giga-MIMO}\label{subsec:gigamimo}
\subsection{NR-Light}\label{subsec:nrlight}
\subsection{Reduced Capability NR}\label{subsec:RedCap}
% Lower Transmit Power for IoT devices
\subsection{Sidelink enhancements}\label{subsec:sidelink}
\subsection{Sleep Modes}\label{subsec:sleep}

\section{Conclusion}\label{sec:conclusion}
% Is the situation relevant to national power consumption?
% Are there effective solutions to reduce power consumption?
% Will this topic become more relevant in the future?


\clearpage

\appendix
% \acrlong{gcd}
% \acrshort{gcd}
% \acrfull{lcm}
\glsaddall
\printnoidxglossary[type=\acronymtype,nonumberlist]

\nocite{*}
\renewcommand*{\refname}{\section{References}}
\bibliography{sources}{}
\end{document}
